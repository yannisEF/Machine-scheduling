%%%%%%%%%%%%%%%%%%%%%%%%%%%%%%%%%%%%%%%%%%%
%%% DOCUMENT PREAMBLE %%%
\documentclass[12pt]{article}
\usepackage[T1]{fontenc}
%\usepackage{natbib}
\usepackage{url}
\usepackage[utf8x]{inputenc}
\usepackage{amsmath}
\usepackage{graphicx}
\graphicspath{{images/}}
\usepackage{parskip}
\usepackage{fancyhdr}
\usepackage{vmargin}
\usepackage{mathtools}
\usepackage{enumitem}
\setmarginsrb{2 cm}{2.5 cm}{2 cm}{2.5 cm}{1 cm}{1.5 cm}{1 cm}{1.5 cm}

\usepackage{blindtext}
\usepackage{tcolorbox}
\tcbuselibrary{minted,breakable,xparse,skins}

\definecolor{bg}{gray}{0.95}
\DeclareTCBListing{mintedbox}{O{}m!O{}}{%
  breakable=true,
  listing engine=minted,
  listing only,
  minted language=#2,
  minted style=default,
  minted options={%
    linenos,
    gobble=0,
    breaklines=true,
    breakafter=,,
    fontsize=\small,
    numbersep=8pt,
    #1},
  boxsep=0pt,
  left skip=0pt,
  right skip=0pt,
  left=25pt,
  right=0pt,
  top=3pt,
  bottom=3pt,
  arc=5pt,
  leftrule=0pt,
  rightrule=0pt,
  bottomrule=2pt,
  toprule=2pt,
  colback=bg,
  colframe=orange!70,
  enhanced,
  overlay={%
    \begin{tcbclipinterior}
    \fill[orange!20!white] (frame.south west) rectangle ([xshift=20pt]frame.north west);
    \end{tcbclipinterior}},
  #3}

\title{Rapport de projet}
% Title
\author{
Yannis Elrharbi-Fleury \\
Yuan Fangzheng
}						
% Author
\date{25/05/2021}
% Date

\setcounter{secnumdepth}{4}

\makeatletter
\let\thetitle\@title
\let\theauthor\@author
\let\thedate\@date
\makeatother

\pagestyle{fancy}
\rhead{\thedate}
\lhead{\thetitle}
\cfoot{\thepage}
%%%%%%%%%%%%%%%%%%%%%%%%%%%%%%%%%%%%%%%%%%%%
\begin{document}
\bibliographystyle{IEEEtran}
%%%%%%%%%%%%%%%%%%%%%%%%%%%%%%%%%%%%%%%%%%%%%%%%%%%%%%%%%%%%%%%%%%%%%%%%%%%%%%%%%%%%%%%%%

\begin{titlepage}
	\centering
    \vspace*{0.5 cm}
   \includegraphics[scale = 0.075]{Images/logo_SU.jpeg}\\[1.0 cm]	% University Logo
\begin{center}    \textsc{\Large   Sorbonne Université}\\[2.0 cm]	\end{center}% University Name
	\textsc{\Large Résolution de problème : problème d'ordonnancement}\\[0.5 cm]				% Course Code
	\rule{\linewidth}{0.2 mm} \\[0.4 cm]
	{ \huge \bfseries \thetitle}\\
	\rule{\linewidth}{0.2 mm} \\[1.5 cm]
	
	\begin{minipage}{0.4\textwidth}
		\begin{flushleft} \large
		\emph{Encadrant :}\\
		Evripidis Bampis\\
		\textbf{\Large }
			\end{flushleft}
			\end{minipage}~
			\begin{minipage}{0.4\textwidth}
            
			\begin{flushright} \large
			\emph{Étudiants :}\\
			\theauthor
		\end{flushright}
           
	\end{minipage}\\[2 cm]
	
\end{titlepage}

%%%%%%%%%%%%%%%%%%%%%%%%%%%%%%%%%%%%%%%%%%%%%%%%%%%%%%%%%%%%%%%%%%%%%%%%%%%%%%%%%%%%%%%%%
\newpage																		
\renewcommand*\contentsname{Table des Matières}
\tableofcontents 

\newpage
%%%%%%%%%%%%%%%%%%%%%%%%%%%%%%%%%%%%%%%%%%%%%%%%%%%%%%%%%%%%%%%%%%%%%%%%%%%%%%%%%%%%%%%%%
\setlength{\parindent}{2ex}

\section{Introduction}
Ce projet porte sur l'étude de solutions au problème d'ordonnancement. \par

Étant donné $N$ tâches et $1$ machine, il s'agit de trouver un ordonnancement de ces tâches minimisant la somme de leur temps de complétude (minimiser le temps d'attente total). \par

La machine ne connaît pas forcément leur durée d'exécution réelle. \par

Dans ce rapport, nous étudions et mesurons la qualité de plusieurs solutions en fonction de l'erreur de prédiction. \\

Le code est écrit en Python et sera fourni en annexe.

\newpage
\section{Structure du programme}

Nous avons naturellement opté pour une approche orientée objet du problème.

\subsection{Les distibutions}
La classe \emph{Distribution} représente un ensemble de distributions de probabilité et leurs paramètres. \\

Lors de son instanciation, elle prend en argument des fonctions permettant de générer un tuple de valeurs. \\

La méthode \emph{sample} renvoie un tuple contenant :
\begin{itemize}
	\item une durée réelle
	\item une durée erreur de prédiction
	\item un instant d'arrivée
\end{itemize}

%% DECRIRE CHOIX DE DISTRIBUTION

\subsection{Les tâches}
La classe \emph{Task} représente une tâche, dont les attributs sont générés à partir d'un objet de type \emph{Distribution}. \\

Elle possède notamment comme attributs :
\begin{itemize}
	\item un ensemble de durée : durée réelle, durée prédite (générées à partir de \emph{Distribution})
	\item un état : \emph{paused, running, finished, not available}
	\item un curseur \emph{currentStep} permettant d'avancer dans l'exécution de la tâche
	\item un numéro d'identification 
\end{itemize}

Une tâche possède trois méthodes :
\begin{itemize}
	\item \emph{hasFinished} : renvoie si la tâche est achevée ou non
	\item \emph{forward} : exécute la tâche d'un pas de temps, renvoie une exécution de \emph{hasFinished}
	\item \emph{restart} : réinitialise la tâche à son état initial
\end{itemize}

\subsection{Les machines}
Notre idée était de créer une classe \emph{Machine} représentant une machine capable de travailler sur un ensemble de tâches. Les différents algorithmes que nous présenterons dans la partie suivante en héritent. \\

Chaque machine possède notamment comme attributs :
\begin{itemize}
	\item des dictionnaires de tâches à différents états : \emph{allTasks, workingTasks, pausedTasks, finishedTasks}
	\item une vitesse d'exécution
	\item une horloge donnant le temps de la machine
	\item une clée d'affichage (une fonction lmabda) permettant de trier les tâches de la machine lors de son affichage
\end{itemize}

Ainsi, \emph{Machine} possède plusieurs méthodes concernant ses tâches : ajouter ou supprimer des tâches; démarrer, mettre en pause ou terminer une tâche. \\

Elle possède aussi des méthodes permettant de les traîter :
\begin{itemize}
	\item une méthode de travail \emph{work} faisant travailler les tâches sur un pas de temps de la machine
	\item une méthode abstraite de traitement \emph{run} traîtant les tâches avant chaque étape de travail, elle permet d'introduire l'algorithme de la machine
	\item une méthode de démarrage \emph{boot} démarrant la machine et la faisant tourner jusqu'à ce que toutes ses tâches soient terminées
\end{itemize}

\subsection{Les machines parallèles}
La classe \emph{Parallel} fonctionne de façon analogue à \emph{Machine}, à ceci près qu'elle ne fournit pas le travail aux tâches elle même. \\

Le travail sur les tâches est effectué par deux machines (\emph{Prediction} et \emph{Round-Robin}) que possède \emph{Parallel}, les exécutant avec une vitesse $\lambda$ et $1 - \lambda$.

\newpage
\section{Implémentation des algorithmes}

Grâce à notre refléxion claire sur les structures de données, nous avons pu facilement implémenter les aglorithmes demandés. \\

Dans cette partie, nous nous contenterons de présenter le fonctionnement de ceux-ci. Nous étudierons leurs performances dans la prochaine section. \\

Les algorithmes suivant héritent de \emph{Machine} ou \emph{Parallel} et ne font que surchager la méthode \emph{run}. \\

\subsection{Algorithme optimal : Shortest processing time}

L'agorithme \emph{SPT} est un algorithme optimal pour ce problème. \\

Il connaît la durée réelles des tâches et les exécute de la plus courte à la plus grande. \\ 

\begin{mintedbox}{python}
    def run(self, step):
        if len(self.workingTasks) == 0:
            nextTask = sorted(list(self.pausedTasks.values()), key=lambda x:x.realLength)[0]
            self.startTask(nextTask)
        return self.work(step)
\end{mintedbox}

Sa méthode \emph{run} est claire : si aucune tâche n'est en cours d'exécution, alors la machine exécute la plus courte.

\subsection{Exécution par prédiction}

Pour l'algorithme \emph{Prediction}, la machine n'a pas accès aux durées réelles des tâches. Elle ne connait que leur durée prédite, avec par conséquent une certaine erreur. \\

\begin{mintedbox}{python}
    def run(self, step):
        if len(self.workingTasks) == 0:
            nextTask = sorted(list(self.pausedTasks.values()), key=lambda x:x.predLength)[0]
            self.startTask(nextTask)
        return self.work(step)
\end{mintedbox}

La méthode est quasiment identique à l'algorithme précédent, à ceci près que les tâches sont triées en fonction de leur durée prédite.

\subsection{Round-Robin}

L'algorithme \emph{Round-Robin} partage son travail de façon égale entre les tâches. \\

Cet algorithme est utile dans le cas où les prédictions sont mauvaises, car il possède un rapport de compétitivité $\max_i {\frac {A(I)} {OPT(I)}}$ de $2$. \\

Notre implémentation se déroule en deux phases : l'initialisation (toutes les tâches démarrent) et l'exécution (\emph{run}). \\

\begin{mintedbox}{python}
    def _initRun(self):
        for task in self.allTasks.values():
            self.startTask(task)
            
    def run(self, step):
        if self.currentTime == 0:
            self._initRun()

        self.speed = self.initSpeed / len(self.workingTasks)
        return self.work(step)
\end{mintedbox}

Il suffit de changer la vitesse de la machine en fonction du nombre de tâches restantes à chaque pas de temps.

\subsection{Exécution parallèle}

Comme vu précédemment, \emph{Parallel} exécute \emph{Predicition} et \emph{Round-Robin} aux vitesses $\lambda$ et $1 - \lambda$. \\

Ainsi, lors de l'initialisation de la machine, on définit ses sous-machines :
\begin{mintedbox}{python}
        self.speed = speed
        self.prediction = Prediction(speed * lmb, key)
        self.roundRobin = RoundRobin(speed * (1-lmb), key)
\end{mintedbox}

Les deux machines possèdent les même tâches en référence, et travaillent ensemble à leur avancement. \\

La méthode \emph{run} prend la forme suivante :
\begin{mintedbox}{python}
    def run(self, step):
        self.currentStep += step

        self.finishTasks()

        if not bool(self):
            self.prediction.run(step)
            self.roundRobin.run(step)

        return bool(self)
        
    def __bool__(self):
        return bool(self.prediction) or bool(self.roundRobin)
\end{mintedbox}

Avec les méthodes \emph{bool} renvoyant si les machines ont fini leur exécution ou non.

\subsection{Exécution parallèle dynamique}

%% A FINIR DE REDIGER

\newpage
\section{Résulats expérimentaux}
\subsection{Cadre classique}
\subsection{Introduction des dates d'arrivée}


\end{document}
